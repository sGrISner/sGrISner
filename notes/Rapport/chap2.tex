\evenchapter[Analyse fonctionnelle et structurelle]{Analyse fonctionnelle et structurelle}

\textit{Le présent chapitre a pour objectif de présenter toutes les fonctionnalités attendues, avec éventuellement la mise en évidence de priorités}

\section{Contraintes pour l'utilisateur}

\begin{itemize}
	\item Modélisation du problème 
	\begin{itemize}
		\item Modèle multiclasse
		\item Modèle multilabel
		\begin{itemize}
			\item Modèle multiclasse choisi pour la première implémentation MAIS flexibilité du programme
			\begin{itemize}
				\item impact sur le formalisme des données en entrée/sortie
				\item impact sur le choix de l'interface à montrer
			\end{itemize}
		\end{itemize}
	\end{itemize}
\end{itemize}

\section{Organisation des données}

\subsection{Données en entrée}

\begin{itemize}
	\item Description des données en entrée
	\begin{itemize}
		\item Fichier avec les classes
		\item Fichier des résultats de la classification
		\item Dossier regroupant les emprises
		\item Ortho-image
	\end{itemize}
	\item Interface de chargement des données
\end{itemize}

\subsection{Données en sortie}

\begin{itemize}
	\item Fichier avec les entités modifiées (ou non)
\end{itemize}

\subsection{Structuration des données}

\begin{itemize}
	\item Diagramme de classe
\end{itemize}



\section{Analyse fonctionnelle}

\subsection{Diagramme des cas d'utilisation (principales fonctionnalités à inclure)}

\subsection{Diagramme d'activité / détail des fonctions}

Détail des entrées et sorties / Description détaillée de la fonction (algo/pseudo-code/ADL) / Références en cas d'utilisation d'algorithmes existants\newline

\begin{itemize}
	\item Lecture des fichiers de classes (=> dictionnaire)
	\item Lecture des fichiers des résultats de la classification (=> matrice)
	\item Stratégies de choix des entités à présenter
	\begin{itemize}
		\item Plusieurs stratégies possibles
		\item Stratégies offline/online
	\end{itemize}
	\item Affichage
	\begin{itemize}
		\item Recherche des bâtiments dans les emprises
		\begin{itemize}
			\item si présence de l'identifiant = lancement du traitement
			\item sinon = popup d'erreur (flexibilité)
		\end{itemize}
		\item Lecture des fichiers .SHP
		\begin{itemize}
			\item calcul de la fenêtre d'affichage
			\item ajout des marges pour calculer les angles repères
		\end{itemize}
		\item Lecture de l'orthoimage
		\begin{itemize}
			\item recherche du point origine
			\item recherche de la taille d'un pixel
			\item calcul des coordonnées des angles de l'emprise en pixels dans l'orthoimage
			\item sélection de la matrice d'orthophoto correspondant et copie dans l'interface
		\end{itemize}
		\item Affichage de l'emprise
		\item Affichage du texte
		\begin{itemize}
			\item Affichage de l'entité en cours et de ses caractéristiques (classe actuelle et probabilité)
			\item Affichage des boutons de choix
		\end{itemize}
	\end{itemize}
	\item Interaction
	\begin{itemize}
		\item OK = passage à l'entité suivante
		\item pas OK
		\begin{itemize}
			\item affichage d'une fenêtre popup (différente selon le modèle)
			\item passage à l'entité suivante
		\end{itemize}
	\end{itemize}
\end{itemize}


\subsection{Conception de l'interface}

MVC

\subsection{Maquette du projet}




